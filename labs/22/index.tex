\documentclass[12pt, a5paper]{book}
\usepackage[utf8]{inputenc}
\usepackage[T2A]{fontenc}
\usepackage{amsmath}
\usepackage{amsthm}
\usepackage{amssymb}
\usepackage{mathtext}
\usepackage[russian]{babel}

% Устанавливаем отступы от краёв страницы
\usepackage{geometry}
%\geometry{papersize={14.8 cm,21 cm}}
\geometry{left=1.6cm}
\geometry{right=1.6cm}
\geometry{top=2cm}
\geometry{bottom=2cm}

% Шаманим межстрочные интервалы, что бы было похоже на оригинал
\renewcommand{\baselinestretch}{.9}
\jot = 1pt

% Шаманим колонтитулы
\usepackage{fancyhdr}
\pagestyle{fancy}
\fancyhead{}
\fancyhead[LE,RO]{\thepage}
\fancyhead[C]{\textit{Глава \thechapter}} 
\fancyfoot{}

% Шаманим заголовки (оформление новой главы)
\usepackage{titlesec}
\titleformat{\chapter}[display]{\filcenter\thispagestyle{empty}}{\textit{Глава \thechapter}\vspace{2em}}{0pt}{\bfseries\MakeUppercase}
\titlespacing\chapter{0pt}{0pt plus 4pt minus 4pt}{24pt plus 2pt minus 2pt}

% Это уже не так обязательно, но есть интересная штука,
% Которая позволит проще работать с теорамами
% Шаманим стили
\newtheoremstyle{indented}
  {.25em}% space before
  {.25em}% space after
  {\itshape}% body font
  {1.25em}% indent
  {\bfseries}% header font
  {.}% punctuation
  {.5em}% after theorem header
  {}% header specification (empty for default)

%Сами окружения теорем, определений и докозательств
\makeatletter
\newenvironment{breakproof}[1][\proofname]{
  \small{Д о к а з а т е л ь с т в о}\@addpunct{.} \ignorespaces}{%
  \@endpefalse}
\makeatother

\theoremstyle{indented}
\newtheorem{theorem}{Теорема}
\newtheorem{definition}{Определение}

% Так как это ЛР, то у меня нет всей книги, 
% Поэтому вручную устанавливаю счетчики: 
% Номер страницы и главы
% И счетчик теорем
\setcounter{chapter}{18}
\setcounter{page}{276}
\setcounter{theorem}{315}
\setcounter{definition}{72}


\begin{document}
    % типо тут могут быть еще главы, а в самом начале титульный лист и т.д
    % пхпхп повезло мне со сменой глав
    \noindent Полагая
$$
    y = g (x)
$$
и принимая во внимание теорему  296, имеем для $0 < |h| < \varepsilon  $
с надлежаще выбранным  $\varepsilon > 0$\ :
\begin{gather*}
    0 = 0 - 0 = f (x + h, g (x + h)) - f (x,g (x)) =\\
    = f (x + h, y + k) - f (x, y) = \\
    = hf_1 (x, g (x)) + kf_2 (x, g (x)) + h\varphi (h) + k\psi (h)
\end{gather*}
, где
$$
    \lim_{ h = 0} \varphi (h) = 0, \lim_{ h = 0} \psi (h) = 0\text{.}
$$
Но
$$
    f_2 (x, g (x)) + \psi (h) \neq 0
$$
для $0 < |h| < \varepsilon_1$ с надлежаще выбранным $\varepsilon_1$,  $0 < \varepsilon_1 < \varepsilon.$
По-\\этому
$$
        \frac{k}{h} = - \frac{f_1 (x, g ) + \psi (h)}{f_2 (x, g (x)) + \psi (h)}
$$
и, следовательно,
$$
        \lim_{h = 0} = - \frac{f_1 (x, g (x))}{f_2 (x, g (x))}\text{.}
$$
\par\textbf{Пример} (снова умышленно старый):
$$
        f (x,y) = x^2 + y^2 -1,\quad |x| < 1,\quad y > 0\text{.}
$$
Здесь
\begin{align*}
        & f_2 (x,y) = 2y > 0,\\
        & f_1 (x,y) = 2x,
\end{align*}
и, следовательно,
$$
        g' (x) = -\frac{2x}{2y} = -\frac{x}{y} = -\frac{x}{g (x)}\text{.}
$$
(И действитвельно, как мы давно уже знаем,
\begin{align*}
    & y = g (x) = \sqrt{1 - x^2},\\
    & g' (x) = -\frac x{\sqrt{1 - x^2}}\text{.)}
\end{align*}
    \chapter{Круговые функции}
\par Речь будет итти о функциях, обратных к тригонометрическим.

\begin{theorem}\label{t_once_sin}
    Для $|x| \leqslant 1$ существует точно одно $y$ такое, что 
   $$\sin y = x,\quad |y| \leqslant \frac {\pi}{2}$$  
\end{theorem}

\begin{breakproof}
    Согласно теореме 278, $\sin y$ монотонно возрастает в 
    $\left\lbrack -\frac{\pi}{2}, \frac{\pi}{2} \right\rbrack$. Так как
    $$\sin\left(-\frac{\pi}{2}\right) = -1,\quad \sin \frac{\pi}{2} = 1\text{,}$$
    то, следовательно в силу теоремы 148, требуемое 
    y существует и однозначно определено.
\end{breakproof}
\begin{definition}
    $\mathrm{arc\ sin}\ x$ для $\;|x| \leqslant 1\;$ есть $y$ из теоремы ~\ref{t_once_sin}.\par
    \textnormal{$\mathrm{arc\ sin}$ читается: арксинус (или аркус синус).}
\end{definition}
\begin{theorem}
    $\frac{d\ \mathrm{arc\ sin}\ x}{dx} = \frac{1}{\sqrt{1 - x^2}} \text{ при  } |x| < 1$.
\end{theorem}

\begin{breakproof}
    Положим 
    $$ y = \mathrm{arc\ sin}\ x,$$
    тогда
    \begin{gather*}
        |y| < \frac {\pi}{2},\\
        x = \sin y,\\
        \frac {dx}{dy} = \cos y > 0.
    \end{gather*}
    и, следовательно, по теореме 313
    $$\frac{dy}{dx} = \frac{1}{\frac{dx}{dy}} = \frac{1}{\cos y} = \frac{1}{\sqrt{1 - \sin^2y}} = \frac{1}{\sqrt{1-x^2}}$$
\end{breakproof}
\end{document}