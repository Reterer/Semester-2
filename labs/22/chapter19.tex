\chapter{Круговые функции}
\par Речь будет итти о функциях, обратных к тригонометрическим.

\begin{theorem}\label{t_once_sin}
    Для $|x| \leqslant 1$ существует точно одно $y$ такое, что 
   $$\sin y = x,\quad |y| \leqslant \frac {\pi}{2}$$  
\end{theorem}

\begin{breakproof}
    Согласно теореме 278, $\sin y$ монотонно возрастает в 
    $\left\lbrack -\frac{\pi}{2}, \frac{\pi}{2} \right\rbrack$. Так как
    $$\sin\left(-\frac{\pi}{2}\right) = -1,\quad \sin \frac{\pi}{2} = 1\text{,}$$
    то, следовательно в силу теоремы 148, требуемое 
    y существует и однозначно определено.
\end{breakproof}
\begin{definition}
    $\mathrm{arc\ sin}\ x$ для $\;|x| \leqslant 1\;$ есть $y$ из теоремы ~\ref{t_once_sin}.\par
    \textnormal{$\mathrm{arc\ sin}$ читается: арксинус (или аркус синус).}
\end{definition}
\begin{theorem}
    $\frac{d\ \mathrm{arc\ sin}\ x}{dx} = \frac{1}{\sqrt{1 - x^2}} \text{ при  } |x| < 1$.
\end{theorem}

\begin{breakproof}
    Положим 
    $$ y = \mathrm{arc\ sin}\ x,$$
    тогда
    \begin{gather*}
        |y| < \frac {\pi}{2},\\
        x = \sin y,\\
        \frac {dx}{dy} = \cos y > 0.
    \end{gather*}
    и, следовательно, по теореме 313
    $$\frac{dy}{dx} = \frac{1}{\frac{dx}{dy}} = \frac{1}{\cos y} = \frac{1}{\sqrt{1 - \sin^2y}} = \frac{1}{\sqrt{1-x^2}}$$
\end{breakproof}